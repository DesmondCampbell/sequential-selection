\documentclass{article}


\usepackage[round]{natbib}
\usepackage{amsmath,amssymb,amsthm,bm,enumerate,mathrsfs,mathtools}
\usepackage{latexsym,color,verbatim,multirow}
\usepackage{graphicx}
\usepackage{caption}
\usepackage{subcaption}
\usepackage{tikz}
\usepackage{geometry}
\usetikzlibrary{shapes,arrows}
\tikzstyle{block} = [rectangle, draw, fill=white!20,
    text width=7em, text centered, rounded corners, minimum height=4em]
\tikzstyle{title} = [text width=7em, text centered, font=\bfseries]
\tikzstyle{line} = [draw, -latex']


\usepackage{mycommands}

\begin{document}

\newtheorem{theorem}{Theorem}
\newtheorem{corollary}[theorem]{Corollary}
\newtheorem{lemma}[theorem]{Lemma}
\newtheorem{observation}[theorem]{Observation}
\newtheorem{proposition}[theorem]{Proposition}
\newtheorem{definition}[theorem]{Definition}
\newtheorem{claim}[theorem]{Claim}
\newtheorem{fact}[theorem]{Fact}
\newtheorem{assumption}[theorem]{Assumption}
\newtheorem{model}[theorem]{Model}

\theoremstyle{definition}
\newtheorem{example}{Example}

\newcommand{\cM}{\mathcal{M}}
\newcommand{\cH}{\mathcal{H}}
\newcommand{\cD}{\mathcal{D}}
\newcommand{\FDR}{\textnormal{FDR}}
\newcommand{\FCR}{\textnormal{FCR}}
\newcommand{\crt}{\phi}
\newcommand{\M}{\mathcal{M}}
\newcommand{\cY}{\mathcal{Y}}
\newcommand{\cX}{\mathcal{X}}
\newcommand{\cV}{\mathcal{V}}
\newcommand{\bX}{\mathbf{X}}
\newcommand{\x}{\mathbf{x}}
\newcommand{\Gv}{\;\;\big|\;\;}
%\newcommand{\cP}{\mathcal{P}}
\newcommand{\proj}{\cP}
\newcommand{\pow}{\text{Pow}}
\newcommand{\sF}{\mathscr{F}}
\newcommand{\cF}{\mathcal{F}}
\newcommand{\sC}{\mathscr{C}}
\newcommand{\hJ}{\widehat{J}}
\newcommand{\bH}{\mathbf{H}}
\newcommand{\bM}{\mathbf{M}}
\newcommand{\hK}{\widehat{K}}
\newcommand{\leqAS}{\overset{\textrm{a.s.}}{\leq}}


\newcommand*\mystrut{\vrule width0pt height0pt depth1.5ex\relax}
\newcommand{\underlabel}{\underbracket[1pt][.5pt]{\mystrut \quad\;\; \sub \quad\;\; }}
\newcommand{\JTcomment}[1]{{\color{blue}{(JT: \bf \sc #1) }}}
\newcommand{\WFcomment}[1]{{\color{red}{(WF: \bf \sc #1) }}}

\title{Adaptive Sequential Model Selection}
\maketitle

\begin{abstract}
  
\end{abstract}



\section{Introduction}

In many data analysis problems, it is difficult to specify in advance a probabilistic model that is simultaneously expansive enough to adequately capture important characteristics of the sampling distribution, and parsimonious enough to yield precise and interpretable inferences. For example, in a linear regression problem, despite having measured thousands of covariates we commonly expect that only a few are actually relevant. That is, if the design matrix is $X\in \R^{n\times d}$, we expect that for some small subset $M\sub \{1,\ldots,d\}$ of the predictor variables, the model
\begin{equation}
  Y \sim N(X_M\beta^M, \sigma^2I_n)
\end{equation}
approximately holds. In this setting, a common goal is to find a small subset of variables that include $M$.

Analysis of variance (ANOVA) tables provide a classical tool for selecting from among a fixed sequence of nested linear models 
\begin{equation}
  \emptyset = M_0 \sub M_1 \sub \cdots \sub \{1,\ldots,d\},
\end{equation}
using $t$- or $F$-tests to compare each $M_k$ to the next larger proposed model $M_{k+1}$.

In exploratory settings it is commonly impossible to specify a sequence of nested models in advance. Procedures abound for adaptively selecting a sequence of nested models, such as forward stepwise regression, LARS, the lasso, and many more \citep{ESL}. However, there are few accepted methods for choosing a final $M_k$ from among the sequence. In particular, using unadjusted $t$- or $F$-tests is no longer a valid option after adaptive selection. For example, if our algorithm begins by selecting the variable most correlated with $Y$, then the $t$-statistic for testing $M_1$ against $M_0$ is very likely to be large, even if $M_0$, the global null hypothesis, is true.

Cross-validation treats $k$ as a tuning parameter and seeks a value $k$ that will produce a good predictive model. However, it does not yield statements of accuracy or inaccuracy for the models selected.

We propose a modular method for selecting a stopping rule based on the selective framework of \citet{fithian2014optimal}. Our method generalizes the ANOVA table to nested sequences of linear or other probabilistic models that are adaptively selected using $Y$. Given data $Y$, a nested model sequence, and the algorithm used to form the sequence, we return a $p$-values $p_k$ for the test at each step $k$. If $M_K$ is the first correct model (note $K$ is random because the sequence is), the resulting $p$-values $p_{K}, \ldots, p_d$ are independent and (stochastically larger than) uniform, conditional on the selected model sequence. By combining our $p$-values with the sequential testing methods of \citet{wager2014}, we can estimate $K$, and control the probability of selecting an index much bigger than the true $K$.

\WFcomment{Put in real data example?}

\WFcomment{Put in non-Gaussian example?}

\section{The Problem of Sequential Model Selection}

\subsection{Notation and Problem Setting}

Assume that we observe data $Y\in\cY$, with unknown sampling distribution $F$. Our goal is to choose a suitable model for $F$ after using some algorithm to propose an adaptive sequence of nested models:
\begin{equation}
  M_0(Y) \sub M_1(Y) \sub \cdots \sub M_d(Y) \sub \cF.
\end{equation}
Specifically, for $k=1,\ldots,d$, we wish to test whether $M_k$ represents a meaningful improvement on $M_{k-1}$; e.g., test whether adding the $k$th selected variable improves the model with the first $k-1$ variables. Let $\bM(Y)$ denote the sequence $(M_0,\ldots,M_d)$, and define its {\em completion index} $K$ to be the index of the first model containing $F$:
\begin{equation}
  K(\bM) = \min \{k:\, F\in M_k\}.
\end{equation}
Let $N_k$ denote the event $\{F\in M_{k-1}(Y)\}\sub \cY$, the event that the null hypothesis is true for the test at stage $k$. That is, if $M_0$ and $M_1$ are false but $M_2$ is true, then $Y\notin N_0$ or $N_1$, but $Y\in N_2 \sub N_3 \sub \cdots$. Note $Y_k\in N_k$ if and only if $k>K$.

For example, suppose that we use the LARS algorithm to build up a linear regression model. Let $J_k(Y)$ be the index of the $k$th variable to have a nonzero coefficient in the LARS path, so that we add variables to the model in the order $X_{J_1}, X_{J_2}, \ldots, X_{J_d}$. It is natural to take model $M_k(Y)$ as
\begin{equation}
  M_k:\, Y \sim 
  \cN\left(\alpha + X_{J_{1}} \beta_{J_{1}} + \cdots X_{J_k}\beta_{J_k}, 
  \;\sigma^2I\right).
\end{equation}
%If the sequence $J_1,\ldots,J_d$ were deterministic instead of adaptive, then %we might construct an ANOVA table for testing whether there was sufficient %evidence to add each variable to the model in sequence. 
If the true mean of $Y$ is linear in $X_{J_1}, X_{J_3},$ and $X_{J_5}$, then $K = 5$ and $Y\in N_k$ for $k=6,\ldots,d$.

Once the algorithm for choosing the sequence $\bM(Y)$ has been specified, we will carry out tests for each comparison. The goal we have set for ourselves is meant to be analogous to sequential testing of ANOVA tables for a fixed sequence of models, though there certainly are other sequential model selection tasks that could be of interest.


\subsection{Stopping Rules}
For our purposes, a {\em stopping rule} $\hK(Y)$ is simply an estimate of the completion index $K$, which we could use to ultimately choose the model $M_{\hK}$ from the sequence.

For our purposes, a good stopping rule is one that does not go on refining the model unnecessarily once it is already adequate; e.g. we might want
\begin{equation}
  \P(\hK > K \gv \bM) \leqAS \alpha, \quad \text{ or } \quad \E\left(\frac{(\hK - K)\vee 1}{\hK \vee 1} \gv \bM \right) \leqAS \alpha
\end{equation}
Although it might initally seem more natural to require $\P(\hK < K \gv \bM) \leq \alpha$ (i.e., that we choose an adequate model with high probability) this goal is typically infeasible as stated. For example, suppose that all the coefficients in a linear regression model are $\beta_j=\ep>0$. As $\ep\to 0$, it becomes impossible to distinguish the true model from the global null model $\beta=0$.

In particular, we will focus on constructing a sequence of $p$-values $(p_1(Y),\ldots,p_d(Y))$. $p_k(Y)$ is a {\em conservative selective $p$-value} if
\begin{equation}
  \P_F\left(p_k(Y) \leq \alpha \gv \bM\right) \leqAS \alpha \text{ on } N_k.
\end{equation}
That is, conditional on the question asked, $p_k$ ought to be stochastically larger than uniform.

We say the sequence $(p_1,\ldots,p_d)$ are {\em jointly conservative selective $p$-values} if
\begin{equation}
  \P_F(p_k(Y) < \alpha_k, \forall k>K \gv \bM) \leqAS \prod_{k>K} \alpha_k, \quad \forall \alpha_1,\ldots, \alpha_d\in (0,1), F\in \cF
\end{equation}
Notice that $K$ is random but the conditional probability above still makes sense because $K$ is a function of $\bM$.

Joint conservatism of $p$-values is a minor generalization of requiring the null $p$-values to be mutually independent given the model sequence $\bM$.

\begin{proposition}
  Suppose that, given $\bM$, the $(p_k)_{k>K}$ are conditionally uniform and mutually independent. Then $(p_k)$ are jointly conservative.
\end{proposition}

A  benefit of having jointly conservative $p$-values is that we can apply the stopping rules of \citet{gsell2013sequential} to control error rates such as the model FDR and model FWER. The next section proposes a generic recipe for obtaining such $p$-values

\subsection{A Sequential Selection Variable for Pathwise FDR}

Define the {\em $k$th pathwise selection variable} as 
\begin{equation}
  S_k(Y) = (M_0,\ldots,M_k,p_1,\ldots,p_{k-1}).
\end{equation}
If for each $k$ the $p$-value $p_k$ is conservative conditional on $S_k$, then the $p$-value sequence $(p_k)$ is jointly conservative. 
\begin{proposition}
  Suppose that for each $k$, $p_k(Y)$ is conservative conditional on $S_k(Y)$ under $F\in M_k(Y)$. Then $(p_k)$ are jointly conservative given $\bM$.
\end{proposition}

Although conditioning on $S_k$ may appear daunting, in many cases it can be a simple matter, as we see next.

\section{Simplifications in Exponential Family Models}

\begin{model}[Exponential Family]\label{mod:exFam}
  Assume that we have finitely many sufficient statistics $T_j(Y)$, and $F$ is in an exponential family after adding enough sufficient statistics to the model:
  \begin{equation}
    Y \sim \exp\left\{\sum_j \theta_j ' T_j(y) - \psi(\theta)\right\} h(y)
  \end{equation}
  Assume we add sufficient statistics to the model one at a time in the order $T_{J_0}, T_{J_1}, \ldots$, so that model $k$ is
  \begin{equation}
    Y \sim \exp\left\{\theta_{J_0} ' T_{J_0}(y) + \theta_{J_1}'T_{J_1}(y) + \cdots + \theta_{J_k}'T_{J_k}(y) - \psi(\cdot)\right\} h(y)
  \end{equation}
\end{model}

Our running Example~\ref{ex:linReg} is a simple case of Model~\ref{mod:exFam}. In that case,
\begin{align*}
  \theta_0 = 1/2\sigma^2, &\qquad T_0(Y) = \|Y\|^2,\\
  \theta_j = \beta_j/\sigma^2, &\qquad 
T_j(Y) = X_j'Y, \text{ for } j>0
\end{align*}

Another example is that of the Wishart, in which we have
\begin{equation}
  Y \sim \exp\left\{ \sum_{ij} \theta_{ij}y_{ij} - \psi(\theta)\right\}h(y)
\end{equation}

To obtain a $p$-value for testing the null hypothesis $M_{k-1}$ against the alternative $M_k \setminus M_{k-1}$, we will typically condition on $U_k=(T_1,\ldots, T_{k-1})$ simply as a way to eliminate $\zeta_k=(\theta_1,\ldots, \theta_{k-1})$ from the problem. In many cases, $p_{k-1}$ is a function only of $U_k$. If so, it is not necessary to condition explicitly on the previous $p$-values.


\section{Simulation: Sequential Model Selection Using the Lasso Path}



\section{Discussion}

We have seen that sequential selection of models --- the 

\bibliographystyle{plainnat}
\bibliography{biblio}

\end{document}
